% Created 2014 by Antonio Arredondo <aarredon@cs.nmsu.edu>
%
% Based off the work of:
%   Sarah Merz <smerz@pacific.edu>  http://www1.pacific.edu/~smerz/Pacific_Beamer_Theme.html
%   Riccardo Murri <riccardo.murri@uzh.ch>  https://github.com/gc3-uzh-ch/beamer-theme-gc3
%
% This tex/latex can be redistributed and/or modified under the terms
% of the GNU Public License Version 2.


%%%%%%%%%%%%%%%%%%%%%%%%%%%%%%%%%%%%                                            
% Department customization                                                      
%                                                                               
% Define your university or organization                                        
\newcommand{\UniName}{New Mexico State University}                              
%                                                                               
% Define your department or group                                               
\newcommand{\UniDeptName}{Computer Science}                                     
%                                                                               
% Define your university or organization logo, by replacing the 'uni_logo' filename 
% with the filename of your image                                               
\newcommand{\UniImage}{\includegraphics[width=2cm]{uni_logo}}                   
%                                                                               
% Optional: If your department or group has logo, uncomment the line below and  
% replace the 'uni_dept_logo' with your filename                                
%\newcommand{\UniDeptImage}{\includegraphics[width=2cm]{uni_dept_logo}}          
 
\documentclass{beamer}
\usetheme{LasCruces}

\title[Short Presentation Title]{~ \\ Long Title:  A sample presentation using \\ a custom beamer theme \\~} 
\author[Short Author Name]{ Antonio Arredondo \\ \UniName \\ \UniDeptName} 
\date{May 01, 2014}

\begin{document}

% title page
\begin{frame}
  \maketitle
\end{frame}

% frame showing all the custom defined boxes
\begin{frame}
  \frametitle{NMSU Colored Boxes}
  \crimsonbox{Crimson Box}{Use the \texttt{\textbackslash crimsonbox} command for this box.}
  \bluebox{Blue Box}{Use the \texttt{\textbackslash bluebox} command for this box.}
  \greenbox{Green Box}{Use the \texttt{\textbackslash greenbox} command for this box.}
  \yellowbox{Yellow Box}{Use the \texttt{\textbackslash yellowbox} command for this box.}
\end{frame}


\begin{frame}
  \frametitle{Bullets}
  \begin{itemize}
    \item These are the default bullets using
    \item \texttt{\textbackslash begin\{itemize\}} command
  \end{itemize} 

  \begin{crimsonitemize}
    \item These bullets are produced with the 
    \item \texttt{\textbackslash begin\{crimsonitemize\}} command
  \end{crimsonitemize}

  \begin{blueitemize}
    \item These bullets are produced with the 
    \item \texttt{\textbackslash begin\{blueitemize\}} command
  \end{blueitemize}

  \begin{greenitemize}
    \item These bullets are produced with the 
    \item \texttt{\textbackslash begin\{greenitemize\}} command
  \end{greenitemize}

  \begin{yellowitemize}
    \item These bullets are produced with the 
    \item \texttt{\textbackslash begin\{yellowitemize\}} command
  \end{yellowitemize}
\end{frame}


\begin{frame}
  \frametitle{Match Bullets}
  \crimsonbox{Color match bullets}{
    \begin{crimsonitemize}
      \item Don't forget to match 
      \item your bullets to the box 
      \item they live in. 
    \end{crimsonitemize}
  }
\end{frame}

\end{document}

